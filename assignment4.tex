\documentclass{amsart}

%\usepackage{etex}
%\usepackage{graphicx}
%\usepackage{amsfonts}
%\usepackage[usenames,dvipsnames]{xcolor}
%\usepackage[bookmarks=true,colorlinks=true, linkcolor=RoyalBlue, citecolor=magenta]{hyperref}
\usepackage{subfig}
\usepackage{mathtools}
\usepackage{amssymb}
\usepackage{dsfont}
\usepackage{cmll}
\usepackage{url}
\usepackage{bbm}
\usepackage{stmaryrd}
\usepackage{tikz}
\usepackage{tikz-cd}
\usepackage{bussproofs}
\usepackage{enumerate}

%Theorem Environments%%%%%%%%%%%%%%%%%%%%%%%%%%%%%%%%%%%%%%%%%%%%%%%%%%%%%%%%%%%%%%%%%%%%%%%%%%%%%%%

\newtheorem{thm}{Theorem}[section]
\newtheorem{lem}[thm]{Lemma}
\newtheorem{cor}[thm]{Corollary}
\newtheorem{prop}[thm]{Proposition}

\theoremstyle{remark}
\newtheorem{rem}[thm]{Remark}

\theoremstyle{definition}
\newtheorem{dfn}[thm]{Definition}
\newtheorem*{notation*}{Notation}

\theoremstyle{definition}
\newtheorem{ex}[thm]{Example}

%Definitions%%%%%%%%%%%%%%%%%%%%%%%%%%%%%%%%%%%%%%%%%%%%%%%%%%%%%%%%%%%%%%%%%%%%%%%%%%%%%%%%%%%%%%%%
\DeclareMathOperator{\aut}{Aut}
\DeclareMathOperator{\List}{List}
\DeclareMathOperator{\irr}{Irr}
\DeclareMathOperator{\lcm}{lcm}
\DeclareMathOperator{\Char}{char}
\DeclareMathOperator{\gal}{Gal}
\DeclareMathOperator{\tr}{Tr}
\DeclareMathOperator{\id}{id}
\DeclareMathOperator{\sgn}{sgn}
\DeclareMathOperator{\colim}{colim}
\DeclareMathOperator{\Ker}{Ker}
\DeclareMathOperator{\im}{Im}
\DeclareMathOperator{\Hom}{Hom}
\DeclareMathOperator{\spa}{span}
\DeclareMathOperator{\Ext}{Ext}
\DeclareMathOperator{\Sym}{Sym}
\DeclareMathOperator{\Res}{Res}
\DeclareMathOperator{\Ind}{Ind}
\DeclareMathOperator{\ad}{ad}
\DeclareMathOperator{\Ob}{Ob}
\DeclareMathOperator{\mat}{Mat}


\def\taking{\colon}
\def\ot{\leftarrow}
\def\too{\longrightarrow}
\def\inj{\hookrightarrow}
\newcommand{\an}[1]{\langle #1 \rangle}
\newcommand{\ceil}[1]{\lceil #1 \rceil}
\newcommand{\floor}[1]{\lfloor #1 \rfloor}
\def\rla{\rightleftharpoons}
\newcommand{\To}[1]{\xrightarrow{#1}}
\newcommand{\From}[1]{\xleftarrow{#1}}
\def\ss{\subseteq}
\newcommand{\ol}[1]{\overline{#1}}
\newcommand{\Id}[1]{\mathbbm{1}_{#1}}
\def\then{\sslash}

\def\op{^{\text{op}}}
\def\Set{\mathbf{Set}}
\def\Rel{\mathbf{Rel}}
\def\Fin{\mathbf{Fin}}
\def\Vect{\mathbf{Vect}}
\def\Pos{\mathbf{Pos}}
\def\Pre{\mathbf{Pre}}
\def\Tot{\mathbf{Tot}}
\def\Mon{\mathbf{Mon}}
\def\Com{\mathbf{Com}}
\def\Grp{\mathbf{Grp}}
\def\Cat{\mathbf{Cat}}
\def\mcC{\mathcal{C}}
\def\mcD{\mathcal{D}}
\def\mcG{\mathcal{G}}
\def\mcL{\mathcal{L}}
\def\bfL{\mathbf{Lin}}
\def\bfW{\mathbf{W}}
\newcommand{\Opd}[1]{\mathcal{O}#1}

\def\BP{\mathbf P}
\def\BQ{\mathbf Q}
\def\PP{\mathbb P}
\def\BB{\mathbb B}
\def\NN{\mathbb N}
\def\ZZ{\mathbb Z}
\def\QQ{\mathbb Q}
\def\RR{\mathbb R}


\newcommand{\pic}[2]{\includegraphics[height=#1]{#2}}

\setcounter{section}{3}

\begin{document}

\section{Assignment 4}

Recall that an adjunction consists of a pair of functors $(L:\mcC\to\mcD,R:\mcD\to\mcC)$, for which we have the following natural isomorphism:
\[\mcD(Lc,d)\cong\mcC(c,Rd)\]
We call $L$ and $R$ the left and right adjoints, and write $L\dashv R$ for shorthand.

\subsection{Problem 1}

We will show that there is a length four sequence of adjunctions
\[\pi_0\dashv \mathbf{D}\dashv U\dashv \mathbf{I}\]
where the \emph{underlying set} functor $U:\Cat\to\Set$ takes a small category $\mcC$ to its object set $|\mcC|$ and the \emph{discrete category} functor $\mathbf{D}:\Set\to\Cat$ map a set $S$ to the category $\ol{S}$ whose object set is $S$ and whose only morphisms are identity arrows.

\begin{enumerate}[(a)]
    \item Prove that $\mathbf{D}\dashv U$. For simplicity, leave one variable fixed and simply prove naturality in the other variable.
    
    \item Give the rule for the behavior of the \emph{indiscrete category} functor $\mathbf{I}:\Set\to \Cat$ so as to satisfy the adjunction isomorphism. You need not check naturality.
    
    \item The \emph{connected components} functor $\pi_0:\Cat\to\Set$ takes a category $\mcC$ and sends it to the set of \emph{connected components} of $\mcC$. More precisely, for $\mcC$-objects $x,y$ let $x\sim y$ if there exists some \emph{zig-zag} of arrows:
    \[x\to t_1 \ot t_2 \to t_3 \ot \cdots y\]
    The idea is that when $x\not\sim y$, $x$ and $y$ are intuitively disconnected.
    We then define $\pi_0\mcC=|\mcC|/\sim$. Show that $\pi_0\dashv\mathbf{D}$. You need not check naturality.
      
\end{enumerate}

\vspace{5 mm}

Recall that a monad $(T,\mu,\eta)$ in a 2-category $\mathfrak{B}$ is a monoid object in the monoidal hom-category $\mathfrak{B}(x,x)$ for some object $x$.

\subsection{Problem 2}

Let $(T:\mcC\to\mcC,\mu:T^2\to T,\eta:\Id{\mcC}\to T)$ be a monad in $\Cat$. Define the \emph{Kleisli category} $\mcC_T$ has object set $|\mcC|$ but arrows defined as:
\[\mcC_T(x,y)=\mcC(x,Ty).\]

The composition of arrows in this category then amounts to mapping a pair of $\mcC$-arrows $f:x\to Ty,g:y\to Tz$ to a $\mcC$-arrow $x\to Tz$, which we define as the following composition:
\[
\begin{tikzcd}
x\arrow[r,"f"] & Ty \arrow[r,"Tg"] & T^2 z \arrow[r,"\mu_z"] & Tz 
\end{tikzcd}
\]

\begin{enumerate}[(a)]
    \item Prove that this composition is associative.
    
    \item Prove that the unit map $\eta_x:x\to Tx$ defines an identity arrow in $\mcC_T$.
    
    \item We secretly use the Kliesli category for the powerset functor $\mathcal{P}:\Set\to\Set$ when solving elementary algebra equations. For example, consider the equation
    \[\sin^2(x) = 1\]
    We first take the preimage of $1$ under $\square^2$ to get $\sin(x) = \{1,-1\}$, and then we apply the preimage of $\sin$ to \emph{each} of the elements in $\{1,-1\}$ and take the union of the two results. Explain how this procedure instantiates composition of arrows in the Kleisli category.
\end{enumerate}

For the following problem, use diagrammatic methods. Feel free to hand-draw your diagrams and then scan them in as an attachment to your email.

\vspace{5 mm}

\subsection{Problem 3}
A dual pair $(L,R)$ in a $2$-category $\mathfrak{B}$ is a pair of $1$-arrows \mbox{$(L:x\to y,R:y\to x)$}, equipped with $2$-arrows
\[\epsilon: L\circ R\to\Id{y} \hspace{10 mm} \eta:\Id{x}\to R\circ L\]
satisfying the zig-zag equations.
\begin{enumerate}[(a)]
    \item Prove that $R\circ L$ is a monad with unit $\eta$ and multiplication given by
    \[\mathbf{1}_R\circ \epsilon\circ\mathbf{1}_L : R\circ L\circ R\circ L \to R\circ L.\]
    \item We say that $(X,\mu,\eta,\delta,\epsilon)$ is a Frobenius monoid if $(X,\mu,\eta)$ is a monoid, $(X,\delta,\epsilon)$ is a comonoid, and the following \emph{Frobenius Law} is satisfied. 
    
    \[\pic{0.8 in}{frobenius_laws.jpg}\]
    
    \vspace{2 mm}
    
    \noindent Prove that $X$ is self dual by constructing dual pair arrows \[I\to X\otimes X\to I\] via composing some arrows in the Frobenius structure, and then showing that they satisfy the zig-zag axioms.
\end{enumerate}

\end{document}