\documentclass{amsart}

%\usepackage{etex}
%\usepackage{graphicx}
%\usepackage{amsfonts}
%\usepackage[usenames,dvipsnames]{xcolor}
%\usepackage[bookmarks=true,colorlinks=true, linkcolor=RoyalBlue, citecolor=magenta]{hyperref}
\usepackage{subfig}
\usepackage{mathtools}
\usepackage{amssymb}
\usepackage{dsfont}
\usepackage{cmll}
\usepackage{url}
\usepackage{bbm}
\usepackage{stmaryrd}
\usepackage{tikz}
\usepackage{tikz-cd}
\usepackage{bussproofs}
\usepackage{enumerate}

%Theorem Environments%%%%%%%%%%%%%%%%%%%%%%%%%%%%%%%%%%%%%%%%%%%%%%%%%%%%%%%%%%%%%%%%%%%%%%%%%%%%%%%

\newtheorem{thm}{Theorem}[section]
\newtheorem{lem}[thm]{Lemma}
\newtheorem{cor}[thm]{Corollary}
\newtheorem{prop}[thm]{Proposition}

\theoremstyle{remark}
\newtheorem{rem}[thm]{Remark}

\theoremstyle{definition}
\newtheorem{dfn}[thm]{Definition}
\newtheorem*{notation*}{Notation}

\theoremstyle{definition}
\newtheorem{ex}[thm]{Example}

%Definitions%%%%%%%%%%%%%%%%%%%%%%%%%%%%%%%%%%%%%%%%%%%%%%%%%%%%%%%%%%%%%%%%%%%%%%%%%%%%%%%%%%%%%%%%
\DeclareMathOperator{\aut}{Aut}
\DeclareMathOperator{\List}{List}
\DeclareMathOperator{\irr}{Irr}
\DeclareMathOperator{\lcm}{lcm}
\DeclareMathOperator{\Char}{char}
\DeclareMathOperator{\gal}{Gal}
\DeclareMathOperator{\tr}{Tr}
\DeclareMathOperator{\id}{id}
\DeclareMathOperator{\sgn}{sgn}
\DeclareMathOperator{\coker}{Coker}
\DeclareMathOperator{\Ker}{Ker}
\DeclareMathOperator{\im}{Im}
\DeclareMathOperator{\Hom}{Hom}
\DeclareMathOperator{\spa}{span}
\DeclareMathOperator{\Ext}{Ext}
\DeclareMathOperator{\Sym}{Sym}
\DeclareMathOperator{\Res}{Res}
\DeclareMathOperator{\Ind}{Ind}
\DeclareMathOperator{\ad}{ad}
\DeclareMathOperator{\Ob}{Ob}
\DeclareMathOperator{\mat}{Mat}


\def\taking{\colon}
\def\ot{\leftarrow}
\def\too{\longrightarrow}
\def\inj{\hookrightarrow}
\newcommand{\an}[1]{\langle #1 \rangle}
\newcommand{\ceil}[1]{\lceil #1 \rceil}
\newcommand{\floor}[1]{\lfloor #1 \rfloor}
\def\rla{\rightleftharpoons}
\newcommand{\To}[1]{\xrightarrow{#1}}
\newcommand{\From}[1]{\xleftarrow{#1}}
\def\ss{\subseteq}
\newcommand{\ol}[1]{\overline{#1}}
\newcommand{\Id}[1]{\mathbbm{1}_{#1}}
\def\then{\sslash}

\def\op{^{\text{op}}}
\def\Set{\mathbf{Set}}
\def\Rel{\mathbf{Rel}}
\def\Fin{\mathbf{Fin}}
\def\Vect{\mathbf{Vect}}
\def\Pos{\mathbf{Pos}}
\def\Pre{\mathbf{Pre}}
\def\Tot{\mathbf{Tot}}
\def\Mon{\mathbf{Mon}}
\def\Com{\mathbf{Com}}
\def\Grp{\mathbf{Grp}}
\def\Cat{\mathbf{Cat}}
\def\mcC{\mathcal{C}}
\def\mcD{\mathcal{D}}
\def\mcG{\mathcal{G}}
\def\mcL{\mathcal{L}}
\def\bfL{\mathbf{Lin}}
\def\bfW{\mathbf{W}}
\newcommand{\Opd}[1]{\mathcal{O}#1}

\def\BP{\mathbf P}
\def\BQ{\mathbf Q}
\def\PP{\mathbb P}
\def\BB{\mathbb B}
\def\NN{\mathbb N}
\def\ZZ{\mathbb Z}
\def\QQ{\mathbb Q}
\def\RR{\mathbb R}


\newcommand{\pic}[2]{\includegraphics[height=#1]{#2}}

\setcounter{section}{1}

\begin{document}

\section{Assignment 2}

\subsection{Problem 1}

For any category $\mathcal{C}$ and object $c\in|\mathcal{C}|$, prove that there is a category, which we denote by $\mathcal{C}/c$ and call the \emph{slice category over} $c$, whose objects are arrows $f:X\to c$ with codomain $c$ and morphisms $[f:X\to c]\to[g:Y\to c]$ are  arrows $\varphi : X\to Y$ for which the following triangle commutes.
\[
\begin{tikzcd}
X \arrow[dr,"f"'] \arrow[rr,"\varphi"] & & Y \arrow[dl,"g"] \\
& c & 
\end{tikzcd}
\]
Analogously, show that there is a category, which we denote by $c/\mathcal{C}$ and call the \emph{slice category under} $c$, whose objects are arrows $f:c\to X$ with domain $c$
and morphisms $[f:c\to X]\to[g:c\to Y]$ are arrows $\varphi : X\to Y$ for which the following triangle commutes.
\[
\begin{tikzcd}
& c \arrow[dl,"f"'] \arrow[dr,"g"] & \\
X  \arrow[rr,"\varphi"'] & & Y 
\end{tikzcd}
\]
Prove that $c/\mathcal{C}$ is the same as $[\mathcal{C}\op/c]\op$.

\subsection{Problem 2}

An arrow $f:X\to Y$ is said to be a \emph{monomorphism} if for all pairs $g,g':W\to X$, the following condition holds:
\[fg=fg' \Rightarrow h=h'.\]
Dually, $f:X\to Y$ is said to be an \emph{epimorphism} if for all $h,h':Y\to Z$:
\[hf=h'f \Rightarrow h=h'.\]
Prove that $f:X\to Y$ in $\Set$ is monomorphism if and only if $f$ is injective, and an epimorphism if and only if $f$ is surjective.

\subsection{Problem 3}

Show that, given a category $\mathcal{C}$, there is a functor \[\mathcal{C}(-,-):\mathcal{C}\op\times\mathcal{C}\to\Set\] given by the rules
\begin{itemize}
    \item $(X,Y)\mapsto\mathcal{C}(X,Y)$ on an object $(X,Y)$
    \item $\mathcal{C}(f,g):\mathcal{C}(X,Y)\to\mathcal{C}(W,Z)::\varphi\mapsto f\then\varphi\then g$ on an arrow $(W\To{f} X,Y\To{g} Z)$.
\end{itemize}

\subsection{Problem 4}

Consider the category $\mathbbm{2}=0\to 1$. Let $F_0,F_1:\mathcal{C}\to\mathcal{D}$ be functors and define the inclusion functors $\mathcal{C}_{i}:\mathcal{C}\to\mathcal{C}\times\mathbbm{2}::X\mapsto (X,i)$ for $i=1,2$. Let $\mathcal{T}:\mathcal{C}\times\mathbbm{2}\to\mathcal{D}$ be a functor such that $\mathcal{T}\circ\mathcal{C}_i=F_i$. Prove that $\mathcal{T}$ gives a natural transformation $F_0\to F_1$. In turn, given a natural transformation $T:F_0\to F_1$, prove that it can be recast as a functor $\mathcal{C}\times\mathbbm{2}\to\mathcal{D}$.

\subsection{Problem 5}

\begin{enumerate}[(a)]
\item Let $(P,\preceq),(Q,\sqsubseteq)$ be preorders. Prove that $\Pre(P,Q)$ is a preorder.

\item If $(P,\preceq),(Q,\sqsubseteq)$ are posets, prove that $\Pos(P,Q)$ is a poset.

\item If $(P,\preceq),(Q,\sqsubseteq)$ are total orders, prove that $\Tot(P,Q)$ is \emph{not} a total order.

\item Show that any set map $X\to P$ is automatically a monotonic map $\mathbf{D}X\to P$.

\item Let $\BB$ have the posetal structure given by the relation $\bot\preceq\top$. Given two maps $\BP,\BQ:\mathbf{D}X\to\BB$, interpreted as predicates (i.e. conditions), describe the logical interpretation of the existence of a natural transformation $\BP\to \BQ$.

\item Describe the join, meet, top, and bottom in the poset $[X;\BB]:=\Pos(\mathbf{D}X,\BB)$.

\item Prove the monotonicity of the map $\iota:\BB\to[X;\BB]::*\mapsto\lambda x.*$.

\item Prove the monotonicity of the map $\exists:[X;\BB]\to\BB::\BP\mapsto\begin{cases}\top & \exists x, \BP(x)=\top \\ \bot &\text{otherwise}\end{cases}$

\item Prove that $(\exists,\iota)$ is a Galois connection.

\item This guarantees that $\exists(\BP\vee \BQ)=\exists \BP\vee \exists \BQ$. Explain why $\exists(\BP\wedge \BQ)\neq \exists \BP\wedge \exists\BQ$.

\item Prove the monotonicity of the map $\forall:[X;\BB]\to\BB::\BP\mapsto\begin{cases}\top & \forall x, \BP(x)=\top \\ \bot &\text{otherwise}\end{cases}$

\item Prove that $(\iota,\forall)$ is a Galois connection.

\item This guarantees that $\forall(\BP\wedge \BQ)=\forall \BP\wedge \forall \BQ$. Explain why $\forall(\BP\vee \BQ)\neq \forall \BP\vee \forall \BQ$.

\item Let $\mathcal{P}:\Set\op\to\Pos$ be the contravariant powerset functor. Construct a natural isomorphism $T:[-;\BB]\To{\cong}\mathcal{P}:\Set\op\to\Pos$.

\end{enumerate}

\iffalse

\subsection{Problem 6}

Consider the arrow category $\Set^\to$.

\begin{enumerate}[(a)]
\item Prove that there are, up to isomorphism, two objects of the form $f:\underline{2}\to\underline{2}$.
\item Describe the isomorphism classes of objects of the form $f:\underline{3}\to\underline{3}$.
\item What pattern do you notice?
\end{enumerate}

\fi

\subsection{Problem 6}

Consider the \emph{dual functor} $\Vect_k(-,k):\Vect_k\op\to\Vect_k$. We write $V^*=\Vect_k(V,k)$ for the \emph{dual space} of $V$. Suppose $V$ has a basis $\mathcal{E}=(\mathbf{e}_i)_{i=1}^n$. 

\begin{enumerate}[(a)]
    \item Show that $V^*$ has a basis $\mathcal{E}^*=(\mathbf{e}^i)_{i=1}^n$ given by $\mathbf{e}^i(\mathbf{e}_j) = \delta_{ij} = \begin{cases} 1 & i=j \\ 0 & i\neq j \end{cases}$
    \item As usual, the basis vectors $(\mathbf{e}_i)_{i=1}^n$, when expressed via the coordinates they induce, can be expressed as standard column vectors:
    \[\mathbf{e}_1 = \begin{bmatrix} 1 \\ 0 \\ \vdots \\ 0 \end{bmatrix} \hspace{5 mm} \mathbf{e}_2 = \begin{bmatrix} 0 \\ 1 \\ \vdots \\ 0 \end{bmatrix} \hspace{2 mm} \cdots \hspace{2 mm} \mathbf{e}_n = \begin{bmatrix} 0 \\ 0 \\ \vdots \\ 1 \end{bmatrix}\]
    Show that the vectors $(\mathbf{e}^i)_{i=1}^n$, when conceived of as linear maps $V\to k$ in the basis $\mathcal{E}$ can be expressed as standard row vectors:
    \[\mathbf{e}^1 = \begin{bmatrix} 1 & 0 & \hdots & 0 \end{bmatrix} \hspace{5 mm} \mathbf{e}^2 = \begin{bmatrix} 0 & 1 & \hdots & 0 \end{bmatrix} \hspace{2 mm} \cdots \hspace{2 mm} \mathbf{e}^n = \begin{bmatrix} 0 & 0 & \hdots & 1 \end{bmatrix}\]
    \item Prove that the linear map $\xi_V:V\to V^*::\mathbf{e}_i\mapsto\mathbf{e}^i$ is an isomorphism for all $V$.
    
    \item Although this is perhaps unintuitive, $\xi_V$ does not extend to a natural transformation $\Id{\Vect_k}\to(-)^*$. Consider the map $L:k^2\to k^2::(a,b)\mapsto(2b,a)$. Show that the following naturality diagram fails to commute.
    \[
    \begin{tikzcd} 
    k^2 \arrow[r,"\varphi_{k^2}"] \arrow[d,"L"'] & (k^2)^* \arrow[d,"(L^*)^{-1}"]\\
    k^2 \arrow[r,"\varphi_{k^2}"'] & (k^2)^*
    \end{tikzcd}
    \]
    \item In contrast, there is a map $\zeta_V:V\to V^{**}::v\mapsto \lambda\varphi.\varphi(v)$ to the \emph{double dual} $V^{**}=\Vect_k(V^*,k)$. Show that this map extends to a natural transformation.
    
    \item Prove that for any $V$, $\ker\zeta_V=0$. Conclude that $\zeta:\Id{\Vect_k}\to(-)^{**}$ is a natural isomorphism. You may assume that $\dim V=\dim V^{**}$.
\end{enumerate}

\end{document}